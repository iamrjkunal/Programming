\documentclass{article}
\usepackage[utf8]{inputenc}

\usepackage{graphics}
\usepackage{subcaption}
\title{MAKE AND GNUPLOT PRESENTATIONS}
\author{Kunal Ranjan \\ 160352}
\begin{document}
\maketitle
\begin{figure}
\centering
\includegraphics{graphs/1_type_a.eps}  \newline
\vspace{5 cm}
 Figure 1: This is a one point scatter graph for 1 thread . \\
 \vspace{0.4 cm}
 \vspace{30 pt}
This graph is One point (scatter) graph for each thread configuration where X-axis is num-of-elements and Y-axis corresponds to execution time for each sample. 
 \end{figure}
 
\begin{figure}
\centering
\includegraphics{graphs/2_type_a.eps} \newline
\vspace{5 cm}
Figure 2: This is a one point scatter graph for 2 threads .\\
\vspace{30 pt}
\vspace{0.4 cm}
 This graph is One point (scatter) graph for each thread configuration where X-axis is num-of-elements and Y-
axis corresponds to execution time for each sample. 
\end{figure}
\begin{figure}
\centering
\includegraphics{graphs/4_type_a.eps}\newline
\vspace{60 pt}
 Figure 3: This is a one point scatter graph for 4 threads .\\
 \vspace{0.4 cm}
 This graph is One point (scatter) graph for each thread configuration where X-axis is num-of-elements and Y-
axis corresponds to execution time for each sample. 
\end{figure}
\begin{figure}
\centering
\includegraphics{graphs/8_type_a.eps}\newline
\vspace{60 pt}
Figure 4: This is a one point scatter graph for 8 threads .\\
\vspace{0.4 cm}
This graph is One point (scatter) graph for each thread configuration where X-axis is num-of-elements and Y-
axis corresponds to execution time for each sample. 
\end{figure}
\begin{figure}
\centering
\includegraphics{graphs/16_type_a.eps}\newline
\vspace{60 pt}
 Figure 5: This is a one point scatter graph for 16 threads .\\
 \vspace{0.4 cm}
This graph is One point (scatter) graph for each thread configuration where X-axis is num-of-elements and Y-
axis corresponds to execution time for each sample. 
\end{figure}


\begin{figure}
\centering
\includegraphics{graphs/type_b.eps}\newline
\vspace{60 pt}
 Figure 6: This is a single line graph for all the threads. \\
 \vspace{0.4 cm}
 A single line graph for each thread configuration where X-axis is num-of-elements and Y-axis is average execution time over 100 samples.
with proper legends for each thread)
\end{figure}
\begin{figure}
\centering
\includegraphics{graphs/type_c.eps}\newline
\vspace{60 pt}
 Figure 7: This is one bar graph.\\
 \vspace{0.4 cm}
 One bar graph with X-axis as number of elements and Y-axis as average speedup (as I have
executed the same configuration for 100 times) w.r.t. one thread execution. Plot the bars representing num of threads for every point in X-axis
\end{figure}
\begin{figure}
\centering
\includegraphics{graphs/type_d.eps}\newline
\vspace{60 pt}
 Figure 8: This is one bar graph with error bars.\\
 \vspace{0.4 cm}
 Plot the same graph as fig 7 but with error bars. Where error bars represent the variance calculated over 100 samples for a particular configuration.
\end{figure}
\end{document}
